\section{導入}
\subsection{\TeX(\LaTeX)とは}
\TeX 「テフ, テック」は, Donald Ervin Knuth氏(以下, Knuth氏)が製作した組版システム\cite{W3C2021}で, 
現在はそれを基にした様々なバージョンが存在する. また, \LaTeX 「ラテフ, ラテック」は
\TeX を基に, マクロパッケージが組み込まれた組版処理システムで, 高品質かつ自由度の高い
組版処理能力と, マクロパッケージに由来する扱いやすさを特徴とする. 
\subsection{組版とは}
組版とは, 原稿及びレイアウト(デザイン)の指定に従って, 文字・図版・写真などを配置する作業の総称. 
\subsection{\LaTeX の特徴・利点}
\LaTeX の特徴として, 先述した通り卓越した組版処理能力, 扱いやすさはもちろんのこと, 
特筆すべきは章番号, 図表番号が自動で振られること, そして数式のデザインがMicrosoft Word
よりも多彩であることである. 後述するコマンドを上手く使えば, あなたが望むままにレポートを
作成することが可能であろう. 
\subsection{コンパイラ}
\TeX ではC言語のようにコードからPDFに書き出す際に\textbf{コンパイラ}によって変換される. 
C言語でもgcc, Visual C++とコンパイラに様々な種類があるように, \TeX でもpLaTeXやLuaLaTeX, 
XeLaTeXのように様々なコンパイラが存在する. 
本書ではフォント等の自由度が高いLuaLaTeXでのコンパイルを前提として説明する. 
基本的には文法は大きく変わらない為, 高速なpLaTeXでのコンパイラも各自で試してみてほしい. 