\section{導入}
\subsection{\TeX(\LaTeX)とは}
\TeX 「テフ, テック」は, Donald Ervin Knuth氏(以下, Knuth氏)が製作した組版システム\cite{W3C2021}で, 
現在はそれを基にした様々なバージョンが存在する. また, \LaTeX 「ラテフ, ラテック」は
\TeX を基に, マクロパッケージが組み込まれた組版処理システムで, 高品質かつ自由度の高い
組版処理能力と, マクロパッケージに由来する扱いやすさを特徴とする. 
\subsection{組版とは}
組版とは, 原稿及びレイアウト(デザイン)の指定に従って, 文字・図版・写真などを配置する作業の総称. 
\subsection{\LaTeX の特徴・利点}
\LaTeX の特徴として, 先述した通り卓越した組版処理能力, 扱いやすさはもちろんのこと, 
特筆すべきは章番号, 図表番号が自動で振られること, そして数式のデザインがMicrosoft Word
よりも多彩であることである. 後述するコマンドを上手く使えば, あなたが望むままにレポートを
作成することが可能であろう. 
\subsection{コンパイラ}
\TeX ではC言語のようにコードからPDFに書き出す際に\textbf{コンパイラ}によって変換される. 
C言語でもgcc, Visual C++とコンパイラに様々な種類があるように, \TeX でもpLaTeXやLuaLaTeX, 
XeLaTeXのように様々なコンパイラが存在する. 
本書ではフォント等の自由度が高いLuaLaTeXでのコンパイルを前提として説明する. 
基本的には文法は大きく変わらない為, 高速なpLaTeXでのコンパイラも各自で試してみてほしい. 
\subsection{プリアンブル}
\LaTeX で文書を作成する際, 本文の前に書かれる部分を\textbf{プリアンブル}と呼ぶ.プリアンブルには,様々な設定を記述することができる.例えば,用紙の大きさやフォントの設定,パッケージの読み込みなどがある.
\subsection{クラス}
\LaTeX ではクラスと呼ばれる概念がある.これは,基本的には何を作るのかによって変わる.
例えば,レポートや論文を書くのであれば\texttt{article}クラスを用いる.
ただ,articleクラスで記述できるのは英語論文であるため,日本語で論文を書きたいならば
\texttt{jsarticle}を用いなければならない.
また,luatexをコンパイラに用いる場合には\texttt{ltjsarticle}クラスが必要になり,
細かい仕様が変わってくる.他にも,本を書くためのbookクラスがあったり,
プレゼン資料を書くためのBeamerクラスがあったり,更に学会によっては
論文のテンプレートに独自のクラスファイルが用いられていることが多い.
本書ではLuaLaTeXを用いるので,記載されているコードは少なくともltjsarticleクラスで
動作する.基本的にはpLaTeXでも動くようなコードが中心だが,
一部動かないコードが存在するので注意が必要.
実際には以下のように用いる.
\begin{lstlisting}
\documentclass[a4paper, titlepage, 10pt]{ltjsarticle}
\end{lstlisting}
a4paperはA4用紙を指定するオプションで,他にもa5paperなどが指定できる.
また,titlepageはタイトルページを作るオプションで,指定しなければ作られない.
10ptは文字の基本サイズを表すオプションで,9pt〜12ptくらいをよく指定する.
\subsection{パッケージ}
\LaTeX は,素の状態で文書を作成しようとしても,自分が作りたいように作成するのは
非常に難しい.そこで,\LaTeX の機能を拡張するための\textbf{パッケージ}が用意されている.目的や用途によってパッケージが分かれており,自分の目的に合ったパッケージを読み込むことで,より効率的,かつ綺麗に文書を作成することができる.使用方法など,詳細は後述する.