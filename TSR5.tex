\section{参考文献の挿入}
参考文献を挿入するときはBibTeXを用いるのが便利である.BibTeXを用いると,文献情報を記述したファイル(.bib)を作成し,それを参照することで,文献リストを自動で生成することができる.
\subsection{BibTeXの使い方}
BibTeXを用いるためには,以下の手順を踏む必要がある.
\begin{enumerate}
  \item 参考文献情報を記述したファイル(.bib)を作成する
  \item \LaTeX ファイル内でBibTeXを読み込む
  \item 参考文献リストを挿入する
\end{enumerate}
\subsection{参考文献情報の記述}
参考文献情報を記述したファイル(.bib)は,以下のような形式で記述する.
\subsubsection{書籍の場合}
\begin{lstlisting}
@book{book1,
  author = "著者名",
  title = "書籍名",
  publisher = "出版社",
  year = "出版年",
}
\end{lstlisting}
\subsubsection{論文の場合}
\begin{lstlisting}
@article{article1,
  author = "著者名",
  title = "論文名",
  journal = "雑誌名",
  volume = "巻数",
  number = "号数",
  pages = "ページ数",
  year = "発行年",
}
\end{lstlisting}

\subsubsection{ウェブサイトの場合}
\begin{lstlisting}
@misc{misc1,
  author = "著者名",
  title = "ページタイトル",
  howpublished = "URL",
  year = "アクセス年月日",
}
\end{lstlisting}
なお,書籍や論文の場合は,必要に応じてvolume, number, pagesを記述する.
\subsection{BibTeXファイルの読み込み}
BibTeXファイルを読み込むためには,以下のコマンドを\LaTeX ファイル内に記述する.
\begin{lstlisting}
\bibliography{参考文献ファイル名}
\end{lstlisting}

\subsection{参考文献リストの挿入}
参考文献リストを挿入するためには,以下のコマンドを\LaTeX ファイル内に記述する.
\begin{lstlisting}
\bibliographystyle{スタイル名}
\bibliography{参考文献ファイル名}
\end{lstlisting}
スタイル名には,jplain, junsrtなどがある.
jplainは参考文献リストをアルファベット順に並べるスタイルである.
一方,junsrtは参考文献リストを引用順に並べるスタイルである.
