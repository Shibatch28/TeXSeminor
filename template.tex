% ドキュメントクラス =================
\documentclass[12pt, a4paper]{ltjsarticle}
% ====================================

% 読み込むパッケージ =================
\usepackage[dvipsnames]{xcolor}         % 色を沢山使えるようにするパッケージ
\usepackage[luatex]{graphicx}           % 画像ファイルを取り込むためのパッケージ
\usepackage{ascmac}                     % 枠やスクリーン環境などを提供するパッケージ
\usepackage{CJKutf8}                    % CJKテキストを使えるようにするパッケージ
\usepackage{luatex85}                   % 古いパッケージに互換性をもたせるパッケージ
\usepackage{multicol}                   % 複数列のレイアウトを実現するパッケージ
\usepackage{physics, amssymb, amsmath}  % 数式に用いる記号や機能を提供するパッケージ
\usepackage{enumitem}                   % 箇条書きのレイアウトを調整するパッケージ
\usepackage{here}                       % 画像や表を強制的に配置するパッケージ
\usepackage{listings, jvlisting}        % ソースコードを載せる用のパッケージ
\usepackage{tikz, tikz-3dplot, braket}  % TikZ用のパッケージ
\usepackage{subcaption}                 % サブキャプションをつけられるパッケージ
% =====================================

% TikZの設定===========================
% 図形描画のための基本ライブラリ
\usetikzlibrary{shapes, shapes.geometric, positioning, through}
% パターン、計算、交差点など、高度な描画機能
\usetikzlibrary{patterns, intersections, calc}
% 図の注釈、角度、矢印などの注記に関連する機能
\usetikzlibrary{quotes, angles, arrows.meta}
% =====================================

% 余白の設定 ==========================
\setlength{\textheight}{\paperheight}
\setlength{\topmargin}{4.6truemm}  % 上の余白を30mm (=1inch[25.4mm] - 4.6mm)に
\addtolength{\topmargin}{-\headheight}
\addtolength{\topmargin}{-\headsep}
\addtolength{\textheight}{-60truemm} % 下の余白も30mmに (TOP+20mm)

\setlength{\textwidth}{\paperwidth}
\setlength{\oddsidemargin}{4.6truemm}  % 左の余白を30mm (=1inch[25.4mm] + 4.6mm)に
\setlength{\evensidemargin}{\oddsidemargin}
\addtolength{\textwidth}{-61truemm} % 右の余白も30mmに
% =====================================

% 図番号などの設定 ====================
\renewcommand{\lstlistingname}{Source Code} % "Listing" を "Source Code" に変更
% 図番号,表番号,式番号に章番号を追加
\numberwithin{figure}{section}
\numberwithin{table}{section}
\numberwithin{equation}{section}
% =====================================

