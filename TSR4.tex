\section{さまざまな「環境」}
\subsection{「環境」とは}
図表を挿入する際,\LaTeX では「環境(environment)」というものを宣言し,その中に図表を挿入する.環境は挿入するものによって分けられており,以下のように対応している.
\begin{description}
  \item[箇条書き]: itemize環境(順番をつけるときはenumerateなど)
  \item[数式]: align環境,equation環境など
  \item[図]: figure環境
  \item[表]: table環境
\end{description}
他にも様々な環境が存在するが,本誌では代表的な環境を紹介する.
また,環境を使用するときは基本的に\texttt{begin}コマンドではじめ,\texttt{end}コマンドで終了する.例えば,itemize環境を用いるときは以下のようになる.
\begin{itembox}[c]{環境の使い方}
  \texttt{
    \hspace{-0.5\zw}\textbackslash begin\{itemize\}\\
    \hspace{2\zw}\textbackslash item アイテムA\\
    \hspace{2\zw}\textbackslash item アイテムB\\
    \textbackslash end\{itemize\}
  }
\end{itembox}
\subsection{箇条書き}
通常の箇条書きにはitemizeを用いる.また,箇条書きする内容の先頭には\texttt{\textbackslash item}とつける必要がある.他にも箇条書きで用語を説明するdescriptionや,数字がつくenumerateも存在する.それぞれの例を以下に示していく.
\newpage
\subsubsection{itemizeの場合}
\begin{itembox}[c]{itemize環境を使う時のソースコード}
  \texttt{
    \hspace{-0.5\zw}\textbackslash begin\{itemize\}\\
    \hspace{2\zw}\textbackslash item アイテムA\\
    \hspace{2\zw}\textbackslash item アイテムB\\
    \textbackslash end\{itemize\}
  }
\end{itembox}
\begin{itemize}
  \item アイテムA
  \item アイテムB
\end{itemize}
\subsubsection{descriptionの場合}
\begin{itembox}[c]{description環境を使う時のソースコード}
  \texttt{
    \hspace{-0.5\zw}\textbackslash begin\{description\}\\
    \hspace{2\zw}\textbackslash item[説明A] アイテムA\\
    \hspace{2\zw}\textbackslash item[説明B] アイテムB\\
    \textbackslash end\{description\}
  }
\end{itembox}
\begin{description}
  \item[説明A] アイテムA
  \item[説明B] アイテムB
\end{description}
\subsubsection{enumerateの場合}
\begin{itembox}[c]{enumerate環境を使う時のソースコード}
  \texttt{
    \hspace{-0.5\zw}\textbackslash begin\{enumerate\}\\
    \hspace{2\zw}\textbackslash item アイテムA\\
    \hspace{2\zw}\textbackslash item アイテムB\\
    \textbackslash end\{enumerate\}
  }
\end{itembox}
\begin{enumerate}
  \item アイテムA
  \item アイテムB
\end{enumerate}
\subsection{数式}
一行で完結する数式,または複数行の数式に一つの式番号を振りたい場合は\texttt{equation}を,複数行の数式にそれぞれ連続した式番号を振りたい場合は\texttt{align}を用いる.なお,式番号を振りたくない場合は「*(アスタリスク)」をequationやalignの直後につける.
\subsubsection{equation}
\begin{itembox}[c]{equation}
  \texttt{
    \hspace{-0.5\zw}\textbackslash begin\{equation\}\\
    \hspace{2\zw}e\textasciicircum\{i\textbackslash pi\}=-1\\
    \textbackslash end\{equation\}\\
    \textbackslash begin\{equation\}\\
    \hspace{2\zw}\textbackslash begin\{split\}\\
    \hspace{4\zw}\textbackslash cos\textasciicircum2\textbackslash theta \& = \textbackslash cos\textasciicircum2\textbackslash theta -\textbackslash sin\textasciicircum2\textbackslash theta \textbackslash \textbackslash\\
    \hspace{4\zw}\& = 2\textbackslash cos\textasciicircum2\textbackslash theta - 1          \textbackslash \textbackslash\\
    \hspace{4\zw}\& = 1 - 2\textbackslash sin\textasciicircum2\textbackslash theta\\
    \hspace{2\zw}\textbackslash end\{split\}\\
    \textbackslash end\{equation\}\\
  }
\end{itembox}
\begin{equation}
  e^{i\pi}=-1
\end{equation}
\begin{equation}
  \begin{split}
    \cos 2\theta & = \cos^2\theta -\sin^2\theta \\
                 & = 2\cos^2\theta - 1          \\
                 & = 1 - 2\sin^2\theta
  \end{split}
\end{equation}
\subsubsection{align}
\begin{itembox}[c]{align}
  \texttt{
    \hspace{-0.5\zw}\textbackslash begin\{align\}\\
    \hspace{4\zw}\textbackslash cos\textasciicircum2\textbackslash theta \& = \textbackslash cos\textasciicircum2\textbackslash theta -\textbackslash sin\textasciicircum2\textbackslash theta \textbackslash \textbackslash\\
    \hspace{4\zw}\& = 2\textbackslash cos\textasciicircum2\textbackslash theta - 1          \textbackslash \textbackslash\\
    \hspace{4\zw}\& = 1 - 2\textbackslash sin\textasciicircum2\textbackslash theta\\
    \textbackslash end\{align\}\\
  }
\end{itembox}
\begin{align}
  \cos 2\theta & = \cos^2\theta -\sin^2\theta \\
               & = 2\cos^2\theta - 1          \\
               & = 1 - 2\sin^2\theta
\end{align}