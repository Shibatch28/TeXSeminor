\section{体裁の調整}
\subsection{ページレイアウト}
余白を調整する方法には,いくつかの方法がある.
ここでは,geometryパッケージを用いた方法とパッケージを用いない方法を紹介する.
\subsubsection{geometryパッケージを用いた方法}
geometryパッケージを用いると,簡単にページレイアウトを調整することができる.
geometryパッケージを用いると,以下のように記述することで,ページレイアウトを調整することができる.
\begin{lstlisting}
\usepackage{geometry}
\geometry{left=3cm,right=3cm,top=3cm,bottom=3cm}
\end{lstlisting}
この場合,左右上下の余白をそれぞれ3cmに設定している.
また,このように指定する方法もある.
\begin{lstlisting}
\usepackage[top=3cm,bottom=3cm,left=3cm,right=3cm]{geometry}
\end{lstlisting}
\subsubsection{パッケージを用いない方法}
パッケージを用いない方法でページレイアウトを調整する場合は,以下のように記述する.
\begin{lstlisting}
% 余白の設定 ---
\setlength{\textheight}{\paperheight}
\setlength{\topmargin}{4.6truemm}  % 上の余白を30mm (=1inch[25.4mm] - 4.6mm)に
\addtolength{\topmargin}{-\headheight}
\addtolength{\topmargin}{-\headsep}
\addtolength{\textheight}{-60truemm} % 下の余白も30mmに (TOP+20mm)

\setlength{\textwidth}{\paperwidth}
\setlength{\oddsidemargin}{4.6truemm}  % 左の余白を30mm (=1inch[25.4mm] + 4.6mm)に
\setlength{\evensidemargin}{\oddsidemargin}
\addtolength{\textwidth}{-61truemm} % 右の余白も30mmに
% ---
\end{lstlisting}
ちなみにこれは本書の設定である.
また,インデントなどの幅を細く変えたいとき,単位を用いる時がある.
このとき,単位として使用できるのは以下の通りである.
\begin{itemize}
  \item mm - ミリメートル.1mmは1ミリメートルに相当する単位. 
  \item cm - センチメートル.1cmは10ミリメートルに相当する単位.
  \item in - インチ.1inは2.54cm(25.4mm)に相当する単位.
  \item pt - ポイント.1ptは1/72インチ(約0.3527mm)に相当する単位.
  \item em - em単位.その場で使われているフォントサイズ(ポイント数)に相当する相対単位.例えば、12ptのフォントなら1emは12pt. 
  \item ex - ex単位.フォントのx高さに基づく相対単位で、x文字の高さを基準とする.フォントによって変動する. 
  \item mu - mu単位.1muは1/18emに相当する非常に小さな単位で、主に数式内の余白調整に使用される. 
  \item zh - 全角文字の高さに基づく単位.CJK(中国語,日本語,韓国語)用の文字サイズに対応. 
  \item zw - 全角文字の幅に基づく単位.
  \item \% - パーセンテージ.例えば,ページやカラムの幅の何\% かを指定する際に使用する.
  \item \textbackslash linewidth - 現在の段落の幅を指定する.段落内での相対的な長さ調整に使える. 
  \item \textbackslash textheight - 本文領域の高さを指定する.ページ内の文章領域のサイズ調整に使われる. 
  \item \textbackslash paperheight - 紙全体の高さを指定する.文書全体のレイアウト設計に使用する. 
  \item \textbackslash paperwidth - 紙全体の幅を指定する.文書全体のレイアウト設計に使用する. 
  \item \textbackslash textwidth - 本文領域の幅を指定する.文章の横幅を設定するための重要な単位. 
  \item \textbackslash columnwidth - カラムの幅を指定する.2カラムや複数カラムのレイアウトで使われる. 
  \item \textbackslash columnsep - カラム間の間隔を指定する.カラム間の余白の調整に使用される. 
  \item \textbackslash columnseprule - カラム間のルールの幅を指定する.カラム間の罫線の太さを調整できる.
\end{itemize}
\subsection{ページスタイル}
fancyhdrパッケージを用いると,ページスタイルを少し変えることができる.
以下のように記述することで,ページスタイルを変更することができる.
\begin{lstlisting}
% ページスタイルの設定 ---
\pagestyle{fancy}
\lhead{}
\chead{}
\rhead{}
\lfoot{}
\cfoot{--\;\thepage\;--}
\rfoot{}
\end{lstlisting}
この場合,ページの中央下部にページ番号が表示される.
本書では,各章で右上に章番号と章名,中央下にページ番号を表示するように設定している.
\subsection{段組}
multicolパッケージを用いると,段組を簡単に設定することができる.
以下のように記述することで,段組を設定することができる.
\begin{lstlisting}
\usepackage{multicol}
\begin{multicols}{2}
  ここに段組したい文章を記述する.
\end{multicols}
\end{lstlisting}
この場合,2つの段組を設定している.
他には,documentclassのオプションでtwocolumnを指定することで,2段組にすることもできる.