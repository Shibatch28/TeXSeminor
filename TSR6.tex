\section{体裁の調整}
\subsection{ページレイアウト}
余白を調整する方法には,いくつかの方法がある.
ここでは,geometryパッケージを用いた方法とパッケージを用いない方法を紹介する.
\subsubsection{geometryパッケージを用いた方法}
geometryパッケージを用いると,簡単にページレイアウトを調整することができる.
geometryパッケージを用いると,以下のように記述することで,ページレイアウトを調整することができる.
\begin{lstlisting}
\usepackage{geometry}
\geometry{left=3cm,right=3cm,top=3cm,bottom=3cm}
\end{lstlisting}
この場合,左右上下の余白をそれぞれ3cmに設定している.
また,このように指定する方法もある.
\begin{lstlisting}
\usepackage[top=3cm,bottom=3cm,left=3cm,right=3cm]{geometry}
\end{lstlisting}
\subsubsection{パッケージを用いない方法}
パッケージを用いない方法でページレイアウトを調整する場合は,以下のように記述する.
\begin{lstlisting}
% 余白の設定 ---
\setlength{\textheight}{\paperheight}
\setlength{\topmargin}{4.6truemm}  % 上の余白を30mm (=1inch[25.4mm] - 4.6mm)に
\addtolength{\topmargin}{-\headheight}
\addtolength{\topmargin}{-\headsep}
\addtolength{\textheight}{-60truemm} % 下の余白も30mmに (TOP+20mm)

\setlength{\textwidth}{\paperwidth}
\setlength{\oddsidemargin}{4.6truemm}  % 左の余白を30mm (=1inch[25.4mm] + 4.6mm)に
\setlength{\evensidemargin}{\oddsidemargin}
\addtolength{\textwidth}{-61truemm} % 右の余白も30mmに
% ---
\end{lstlisting}
ちなみにこれは本書の設定である.
\subsection{ページスタイル}
fancyhdrパッケージを用いると,ページスタイルを少し変えることができる.
以下のように記述することで,ページスタイルを変更することができる.
\begin{lstlisting}
% ページスタイルの設定 ---
\pagestyle{fancy}
\lhead{}
\chead{}
\rhead{}
\lfoot{}
\cfoot{--\;\thepage\;--}
\rfoot{}
\end{lstlisting}
この場合,ページの中央下部にページ番号が表示される.
本書では,各章で右上に章番号と章名,中央下にページ番号を表示するように設定している.
\subsection{段組}
multicolパッケージを用いると,段組を簡単に設定することができる.
以下のように記述することで,段組を設定することができる.
\begin{lstlisting}
\usepackage{multicol}
\begin{multicols}{2}
  ここに段組したい文章を記述する.
\end{multicols}
\end{lstlisting}
この場合,2つの段組を設定している.
他には,documentclassのオプションでtwocolumnを指定することで,2段組にすることもできる.