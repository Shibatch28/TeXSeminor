\section{\LaTeX の基本・テキスト編}
\subsection{パッケージ}
\subsubsection{パッケージとは・パッケージの使い方}
\LaTeX の最大の特徴はパッケージによって多彩な機能を追加できる点である.ではパッケージとはどういうものかというと,いくつかの機能をまとめて使えるようにした,いわばお道具箱のようなものである.例えば,テキストに色をつけたい時,\texttt{xcolor}パッケージを
読み込めばそれを使うことができる.どのように読み込めば良いかというと.\\
\texttt{
\indent
\textbackslash usepackage\{xcolor\}
}\\
とプリアンブルに書くことで読み込むことができる.
\subsection{タイトルの表示}
\subsubsection{タイトル,著者,執筆日の設定}
\LaTeX で文書を作成する際には,タイトル,著者,執筆日を設定する必要がある.
タイトルを設定するには,\texttt{title}コマンドを使用する.著者を設定するには,\texttt{author}コマンドを使用する.執筆日を設定するには,\texttt{date}コマンドを使用する.
\begin{lstlisting}
  \title{タイトル}
  \author{著者}
  \date{執筆日}
\end{lstlisting}
\subsubsection{タイトルの表示}
タイトル,著者,執筆日を設定した後,\texttt{document}環境内で\texttt{maketitle}コマンドを使用することで,それらを表示することができる.
\begin{lstlisting}
\begin{document}
  \maketitle
\end{document}
\end{lstlisting}
\newpage
\subsubsection{document環境とは}
\texttt{document}環境は,\texttt{begin}コマンドと\texttt{end}コマンドで囲まれた部分を指す.\texttt{document}環境内に本文を記述する.
ここに記述された内容が実際に文書として表示される.逆に,\texttt{document}環境外に記述された内容は,文書として表示されずにエラーが起きる.
\subsection{目次}
\subsubsection{目次の表示}
目次を表示するには,\texttt{tableofcontents}コマンドを使用する.\texttt{clearpage}コマンドを合わせて
使用することで,目次を新しいページに表示することができる.
\begin{lstlisting}
  \tableofcontents
  \clearpage
\end{lstlisting}
\subsection{章立て}
\subsubsection{章立ての方法}
レポートにおいて,章立ては必須である.章立てをする際には以下のタグを用いる.
\begin{itemize}
  \item \ttfamily \textbackslash section
  \item \textbackslash subsection
  \item \textbackslash subsubsection
  \item \textbackslash paragraph
\end{itemize}
\texttt{section}は行った実験ごとに章を分ける場合に使用し,\texttt{subsection}はその実験の各項目(目的,実験方法など)
を分けるのに使用する場合が多い.\texttt{subsubsection}に関しては更に細かく章を分けたいときに使用する.\texttt{paragraph}は,
更に細かい章分けに用いる.
\newpage
例えば,工学部2年後期から始まる実験では,数日に分けて実験を行う場合が多いので,以下のように章立てをするのが良いだろう.
\begin{itembox}[c]{章立ての例}
  \ttfamily
  \textbackslash section\{1日目 実験内容\}\\
  \textbackslash subsection\{実験目的\}\\
  $\vdots$\\
  \textbackslash subsection\{考察\}\\
  \textbackslash section\{2日目 実験内容\}\\
  \textbackslash subsection\{実験目的\}\\
  $\vdots$
\end{itembox}
学部4年生以上は,所属する研究室や論文を提出する学会のルールに従うこと.
\subsection{文字の装飾}
論文やレポートなどを執筆したいとき,\textbf{太字}や\textit{Italic},\color{red}色付き文字\color{black}などを使って強調したいことがあるだろう.そこで\LaTeX で使える文字のスタイライズコマンドを以下に示す.
\subsubsection{太字}
太字を挿入したいときは,\texttt{textbf}コマンドを使用する.具体的には,次のように使う.
\begin{itembox}[c]{太字の例}
  \ttfamily
  \textbackslash textbf\{太字にしたい文\}
\end{itembox}
\subsubsection{Italic(斜体)}
斜体に関しては,日本語フォントに斜体が組み込まれていないため,基本的には日本語の斜体はサポートされていない.正確には全くできないというわけではないが,複雑かつ体裁が崩れやすいため,本誌では紹介しない.
英語に関してはシンプルな手法でできるため,以下に斜体にするためのコマンドを示す.
\begin{itembox}[c]{斜体の例}
  \ttfamily
  \textbackslash textit\{斜体にしたい英文\}
\end{itembox}
\subsubsection{等幅}
ソースコードを一部示すときなど,一時的に等幅フォントを使用したい場合は\texttt{texttt}コマンドを使用する.具体的には,次のように使う.
\begin{itembox}[c]{等幅の例}
  \ttfamily
  \textbackslash texttt\{等幅にしたい文\}
\end{itembox}
\subsubsection{色付き文字}
テキストの一部に色をつけたい場合は,\texttt{color}コマンドを使用する.なお,使用できる色については読み込むパッケージに依存しており,xcolorパッケージではさまざまな色が使える.逆にパッケージを読み込まなければ多彩な色付き文字を使うことはできないので,冒頭でxcolorパッケージを読み込まなければならない.
具体的には,次のように使う.
\begin{itembox}[c]{色付き文字の例}
  \ttfamily
  \textbackslash usepackage[dvipsnames]\{xcolor\}
  \% xcolorパッケージをdvipsnamesで読み込み(これで多彩な色を使える)
  \\
  中略
  \\
  \textbackslash color\{色\}色付きにしたい文\textbackslash color\{black\}
\end{itembox}
xcolorパッケージで使用できる色については,OverLeafのドキュメントを参考にすると良い.(\url{https://ja.overleaf.com/learn/latex/Using_colors_in_LaTeX})