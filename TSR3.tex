\section{\LaTeX の基本}
\subsection{章立て}
\subsubsection{章立ての方法}
レポートにおいて,章立ては必須である.章立てをする際には以下のタグを用いる.
\begin{itemize}
  \item \ttfamily \textbackslash section
  \item \textbackslash subsection
  \item \textbackslash subsubsection
  \item \textbackslash paragraph
\end{itemize}
\texttt{section}は行った実験ごとに章を分ける場合に使用し,\texttt{subsection}はその実験の各項目(目的,実験方法など)
を分けるのに使用する場合が多い.\texttt{subsubsection}に関しては更に細かく章を分けたいときに使用する.\texttt{paragraph}は,
更に細かい章分けに用いる.\\
例えば,工学部2年後期から始まる実験では,数日に分けて実験を行う場合が多いので,以下のように章立てをするのが良いだろう.
\begin{itembox}[c]{章立ての例}
  \ttfamily
  \textbackslash section\{1日目 実験内容\}\\
  \textbackslash subsection\{実験目的\}\\
  $\vdots$\\
  \textbackslash subsection\{考察\}\\
  \textbackslash section\{2日目 実験内容\}\\
  \textbackslash subsection\{実験目的\}\\
  $\vdots$
\end{itembox}
\newpage
\subsection{章立ての例}
各学部,学科によって細かいレポートの書式があるため,ここですべてを網羅することはできない.よっていくつかの例を以下に示す.\vskip .5\baselineskip
【電気系のレポートの場合】\cite{IEE}\\
一般社団法人 電気学会."原稿の書き方|一般社団法人 電気学会".原稿の書き方.\url{http://www.iee.jp/tech_mtg/howto/},2023/10/08\vskip .5\baselineskip
【情報系のレポートの場合】\cite{IPSJ}\\
一般社団法人 情報処理学会."LaTeXスタイルファイル、MS-Wordテンプレートファイル 情報処理学会".LaTeXスタイルファイル、MS-Wordテンプレートファイル.\url{https://www.ipsj.or.jp/journal/submit/style.html},2023/10/08\vskip .5\baselineskip
【機械系のレポートの場合】\cite{JSME}\\
一般社団法人 日本機械学会."Japanese-Template-mihon.pdf".日本機械学会論文集.\url{https://www.jsme.or.jp/publish/transact/Japanese-Template-mihon.pdf},2023/10/08\vskip \baselineskip

これらは学会への論文投稿の為のテンプレートであるが,報告書(レポート)として作成する際も基本的に同様の形式で良い.